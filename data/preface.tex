\chapter{序言}
\label{chap:preface}

笔者起初只是想做丁家庄城中村摄影。随着摄影项目的开展和瓶颈,社会学(尤其是批判实
践社会学)便进入了笔者的视野。在对社会学的学习和思考过程中,这个摄影项目逐步发展
为这本免费开源的电子书。本书也可理解为笔者个人三观的思考、探索和总结。

焦虑、质疑、批判是人类进步的必要要素。不能认可批判的建设性价值,将面对停滞、倒退
或向更为危险的境地发展。批判要建立在对真实和事实的求证基础上,它不止有批评,同样
包括对好方面的肯定。我们生活在一个资本现代性的理性愈加强大的时代,它强大和贪心到
试图去规训,并能够在相当程度上规训其它一切理性,妄图称王使所有理性臣服。这更需要
我们具有批判精神。

这本电子书是个四不像但又什么都有的怪物,涉及政治、社科、经济、人文等诸多方面,贯
穿始终的核心是笔者自身应用社会学对于社会和人\improve{以后加入资本?}的思考。
笔者并非学富五车的专业人士,自身缺陷与恶习比比皆是,这均使本书内容存在各种各样缺
陷,惹人耻笑。它也几乎不会产生任何影响力或激起什么波澜。但我想它最重要和最强大的
意义,是展现出一种个人真诚、直接和积极的社会学历程(详情请
见\cref{chap:gerenshehuixue}),一些人也正自觉或不自觉地走在这样的道路上。笔者愿作
这条道路上的一块铺路石,以使同行者不那么孤单、寂寞。这条道路的发展将使人类和社会
受益。

十多年前,笔者在尼采某本书\footnote{年代久远,我已忘了这本书的名字和具体内容,只
  是粗略记忆。}中看到某夫人对她的儿子说“亲爱的,你总是做傻事,做傻事让你特别快
乐”时被击中了。笔者想做好事,想把好事做好,但总阴差阳错的走向相反面,许是无奈之
下只好从这许多傻事中吸取快乐了。十多年过去,笔者已近不惑之年,还是一直在做傻事,
平添父母、妻子、子女的烦恼,或许写作这本书只不过是笔者所做的又一件傻事而已,整件
事只是笔者自我逃避的又一个山丘……


% 最要感谢的是我的妻子韩康利。结婚八年来,
% 我不务正业,不事生产,屡屡做些傻事惹人耻笑,没有经济建树又身在外地。但韩康利虽不
% 支持但也不反对,事实上纵容我傻乐,并默默承担起家中老人与两个孩子的繁重照顾工作。
% 此生我最大的幸运是娶到了韩康利,因这一幸运我不再有任何一点立场指责上天待我凉薄。

本书采用GNU开源协议,源代码放置在\url{https://gitee.io/sd44/dingjia},已经编译好的草
稿可于“草稿PDF文档”文件夹下下载。

% 笔者所见所知并无多少创新、发展,尤其与一些富有强烈人文关怀的专业社科人士相比,我
% 简直像幼儿园小朋友一样所知甚少。但这此书最主要的目的和意义,并非是书中蕴含多少真
% 知灼见,而是在书外。在于本书读者几何,影响力
% 多少,甚至有多少错误其实已经不重要了。变,那就是自我抒发我个人对于社会、国家和政
% 治的见解。至于

% 个人朝三暮四,贪乐误事,丁家庄这个项目常常面临夭折和停滞,其目的和方向也总是
% 一片混沌。

% 在国际政治的角斗场中,如果中国因为社会剧痛剧变从而变得贫弱多病,就必然被许多国家
% 暂时搁置他们彼此之间的争斗,转头立刻联合起来对中国群起攻之,分而食之,进而敲骨吸
% 髓,让中国万劫不复。看不到这一点的人,在政治上是幼稚无知或者是别有用心的。之所以
% 会造成这种局势,并非是因我们中国和外国的政治体制或意识形态不同,实质上我认为我们
% 与他国的共同点——即使在政体和意识形态上——远比不同点多的多,甚至呈多倍比例。只是因
% 为中国物产丰富,疆域辽阔,市场第一,发展态势迅猛及潜力巨大,且在历史上我们同欧美
% 各国的交流,没有他们彼此之间的互通交流频繁。我要声明,我坚决拥护党和国家的领导,
% 对所谓西方自由民主等陷阱和中国重改良等坚决排斥。但是我坚持轻改良,所谓正能量,不
% 是不说难听的话、不指出事物缺陷,而是使事物向更好方向发展的力量,改良是需要的。

% 在这过程中,在项目进行过程中,喜闻有两位姑娘,可能是大学生也在进行丁家庄的社会学
% 问卷调查工作,同意接受调查的村民或租户可以获得50元奖励。我感觉可能是国立大学调查,
% 很是欣喜,希望中国能在社科方面快速发展。

% 如果有可能的话,我想做的下一个项目是不像丁家庄项目这么宏大难控的,它的表达方式更
% 加具体和容易引起社会影响,同时也更加注重感官而不是理性,可以更多借助摄影等方式,
% 那就是关于ADHD(小儿多动症)的批判。小儿多动症是一个症候群,其中一些表现可以归为
% 个人特质或非精神性病变等,国外已有多部相关著作,深度理论构建应当可以从反精神病学
% 与精神病批判上汲取养分(即使没有深度理论,项目应当也可获得成功)。学习社科这一年
% 多来,我已经有爱上社科了,但就现实情况来看,丁家庄项目是我进行的第一个社科项目,
% 也很可能是最后一个。我的年龄、基础、精力、家庭、经济能力等似乎很难允许我再做类似
% 的工作。如果有读者能进行ADHD这个项目就太好了。

% 特别感谢泰安的李玉刚,与我进行多次最接地气的激烈讨论并直言不讳。特别感谢深圳的邱
% 文,在项目初期陷入停滞时,建议我采用人类学的方式去观察丁家庄,并在项目进行过程中
% 多次对我鼓励和指导。特别感谢成都的saintjoe,多次与我就摄影和社会展开探讨,也建议
% 大家多关注一直在进步的他的摄影作品。特别感谢丁家庄原住户朱琳琳,也一直鼓励和建议
% 我,并提出宝贵意见。

% 感谢丁家庄愿意信任我并接受调查的人们。
% 感谢微信上广州HiFi_Tam所建摄影群,其中北京刘烜超,广州Hifi_Tam,广州邱邱,南通兽
% 无不摄,上海Keith,厦门Resean均对本项目不成气,但我也无力去更改的摄影部份提出宝
% 贵意见(以上排名均安拼音顺序,不分先后)。


% 谨以此书献给我的妻子韩康利。

% 2014年,笔者的工作和住宿地点均变更为济南市化纤厂路,距离丁家庄菜市场不
% 足500米。2016年7月份,笔者为练习街头摄影常在街头漫无目的地游荡,有次穿过菜市场忽
% 然发现了一片新景象——丁家新村,济南人俗称丁家庄,这是一个城中村。笔者虽已在济南工
% 作9年,在丁家庄附近生活2年,也去过济南市诸多地方,但从不知道有丁家庄城中村这样一
% 个存在。

% 虽然笔者就出生和成长在一个贫困县,也去过几次农村,但初入丁家庄仍因它表面的破败和
% 杂乱而产生一种恐惧感,总感觉可能有治安危险。它既不同于城市也不同于农村,一些地方
% 甚至比农村还要显得困窘残破。

% 如何不流于表面地展示这里?如何避免消费苦难,去真正深入地表现这个地方?笔者二三十
% 次进入丁家庄城中村,仍未找到答案。生活在深圳的邱文向我提了一个建议,用人类学的眼
% 光去拍摄、组织照片,直接展示他们的衣食住行、教育娱乐等方面,使其呈现出丁家庄城中
% 村的整体生活面貌,并可作为一个样本进行留存。

% 笔者在邱文的建议的基础上,最终选择了社会学方向来做丁家庄的全面考察。“空间生
% 产”这一块比较知名的学者有列斐伏尔、大卫·哈维等人,这些学者基本都是批判实践社会学
% 方向。批判社会学的奠基人一般被认为是马克思,列斐伏尔、大卫·哈维等人也深受马克思主
% 义影响,要较好理解他们的著作,难以跳过马克思。因此我又学习了《资本论》等马克思、
% 恩格斯著作,这使本书带有马克思色彩。

% 社会学学习的进展,和笔者摄影水平的粗浅,笔者渐感有必要
% 随着学习的深入,也因笔者摄影技术的粗浅,笔者渐感摄影的无力,
% 当代世界,不管是左派还是右派,大多数人对马克思的理解常常是肤浅甚至错误,“马克
% 思”往往成为一种标签和符号,一些人看到这个标签和符号就产生强烈的好恶感和价值判定。
% 希望大家能够从理论内容本身来判定,而非这种先入为主。

% 保有随着拍摄练习的深入,我对丁家庄开
% 始关注起来,想将丁家庄城中村作为一个摄影项目来做。在进行过程中,项目的施行方式和
% 角度也始终在变化,从起初单纯个人的摄影项目,发展到对丁家庄城中村的了解欲望,并进
% 一步发展至对中国新型城镇化规划的关注,最终发展至对中国发展的态势见解。