\chapter{新自由主义札记}
\label{chap:neoliber}

本章因是札记性质,可能作为笔者开源免费电子书的一部分出现。

\section{新自由主义的概念和实质宗旨}

Newliberalism,也可称为\textbf{社会自由主义}(Social liberalism),与John
Ruggie在1982年提出的镶嵌型自由主义、嵌入式自由主义(Embedded liberalism)相通。其
代言人为凯恩斯、罗尔斯和德沃金等,主张国家干预经济生活,可通过加大政府支出、投资
来解决失业和消费不足经济危机,重视社会福利。它允许社会和经济的不平等,但主张政治
自由权的平等优于经济自由权的平等,要求国家给予民众关怀和尊重。\cite{newneo}

中国语境中常说的“\textbf{新自由主义}”实为Neoliberalism,即“\textbf{新古典自由
  主义}”,它不同于Newliberalism。维基英文版对Neoliberalism的解释如下:
\begin{quotation}新古典自由主义,主要是指19世纪初与\textbf{自由放任的经济自由主
    义}相关的思想\footnote{古典自由主义(Classical liberalism)中的经济自由\textbf{部分}}在20世纪的复苏。这些想法包括经济自由化的一系列政策,如私有化,财政
  紧缩,放松管制,自由贸易和减少政府支出,以增加私营部门在经济和社会中的作用。这些
  以市场为基础的思想及其所激发的政策促成了从战后的凯恩斯主义(1945-1980)到新古典
  自由主义的范式转变。

  学者现在倾向于将其与朝圣山学派的经济学家弗里德里希·哈耶克、米尔顿·弗里德曼和詹姆
  斯·M·布坎南,以及玛格丽特·撒切尔,罗纳德·里根和艾伦·格林斯潘等政治家和政策制定者
  联系起来。
\end{quotation}

新自由主义理论内部并非铁板一块,充斥着各种学派,如现代货币学派、理性预期学派、供
给学派等等,这些学派彼此之间也存矛盾、异议,各国政府的新自由主义实践依时间、国情
等的不同,对这些理论的侧重均有不同。\cite{neoxuepai}其中一些早期的热情拥护者和参
与者如今也都转向批判立场。\improve[inline]{哈耶克也有转向和反思,我忘记了在哪篇资料
  上,日后增补。}

左翼对新自由主义的批判各有不同见解,但似无本质区别。
\begin{quotation}
  以中国社会科学院“新自由主义研究”课题组撰写的“新自由主义研究”为
  例。该课题组将“新自由主义”的主要观点归纳和概括为以下三点:在经济理论方面大力宣
  扬“三化”(自由化、私有化、市场化),在政治理论方面特别强调和坚持“三否定”(否定
  公有制、否定社会主义、否定国家干预),在战略和政策方面“极力鼓吹以超级大国为主导
  的全球经济、政治、文化的一体化,即全球资本主义化”。\cite{newneo}
\end{quotation}

因笔者水平粗浅,本篇只以大卫·哈维《新自由主义简史》一书为理论中心展开论述。但笔者
认为,有必要提下克里斯·哈曼。

克里斯·哈曼\cite{chrisharmanneo1} \cite{chrisharmanneo2}和大卫·哈维都认为在新自由
主义国家的实践过程中均出现了与它所宣称的背离!在美国、中国等国的新自由主义实
践\footnote{在克里斯·哈曼所著《Theorising neoliberalism》的中译本《对新自由主义理
  论研究的反思》中删去了“中国”。}中,均采用了新自由主义所明确反对的\textbf{凯恩
  斯主义的国家干预}为经济发展提供保障,例如政府大规模财政赤字、不良银行债券资助基
础设施建设和固定资本投资等。而被灌输并实行较为彻底新自由主义化的落后国家,则被美
国等国利用资本全球化进行掠夺和积累。

两人的不同认识为:哈维认为这种国家干预是新自由主义国家实践过程中的\textbf{悖论},
这一悖论所产生的真正原因是“新自由主义的主要实质成就不是生产财富和收入,而是
\textbf{对财富和收入进行分配}\pagescite[][165-166]{davidneoliber}”,
“\textbf{旨在重建资本积累的条件并恢复经济精英(资产阶级)的权
  力}\pagescite[][19-20]{davidneoliber}”。而哈曼以更为激进和传统的马克思主义视角
对哈维提出了批评。哈曼认为“新自由主义”这一意识形态提法具有\textbf{政治模糊性},
它弱化了国家支出在新自由主义实践中的地位,抬高了积累过程中“暴力”的地位,降低了
资本在阶级生产方面的强大理性,弱化了工人阶级的力量。
\begin{quotation}
  由于这些原因,“新自由主义”实际上并不是对今日资本运行的精确描
  述。我们没有面临向\textbf{自由市场资本主义}的回归,这种资本主义在一个世纪以前就
  \textbf{完结}了。我们面临的是这样一个体系,它尝试着在全球范围内重建它体系的各个
  单元来解决它自身的问题,这些单元出现于20世纪的进程中,马克思主义者称之为
  “\textbf{垄断资本主义}”、“\textbf{国家垄断资本主义}”或“\textbf{国家资本主
    义}”。\textbf{国家继续扮演重要角色},想方设法为垄断资本提供便利或进行管理,即使
  生产的国际化使得这样做比战后几十年更加困难。
\end{quotation}

笔者认为,两者可以批判结合起来看。哈维着重资产阶级本身在资本主义中的力量,“新自
由主义”的提法相对微观,更具有时代思潮的针对性,并可能确实有减轻现实压力的考虑,
但是社会各界对这一提法的简述或简单理解往往忽略了哈维对新自由主义实践
中“\textbf{悖论}”的论述。哈曼着重国家在资本主义中的力量,“国家垄断资本主义”相
对宏观,能清晰解释整个实践过程中的异动,赋予阶级以更强力量。克里斯·哈曼
的论述在资本对国家的反制上缺乏力度,在一些地方确实也苛责哈维。两者的结合将使我们
建立更为完整的认识。

\section{哈维对新自由主义实践的具体批判}

新自由主义理论口号所宣称的实则是一种\textbf{乌托邦}。它宣称市场放任的经济自由远比一切
其他方面的自由(尤其是政治自由)更为重要。这种经济自由是其它自由的唯一基础。

至于其实践中实际暴露出来的问题,笔者根据哈维对新自由主义的批判,加以思考并结合其
他资料整理如下:
\begin{enumerate}
\item 美国为首的发达国家向其它弱国推行新自由主义,倡导经济自由,实则是借弱国的新
  自由主义化,用被强国实力所影响的资本全球化相关机构、机器实现自身利益。在收割过
  程中,以强国国力作为背书,\textbf{实现债权强国的近乎无风险与超额利润和绝对主导,
    造成债务弱国的社会悲剧深渊}\footnote{即使是Classic liberalism也反对债权人的近
    乎无风险。}。正如斯蒂格利茨的讽刺“这是多么古怪的世界啊,反倒是贫穷的国家在补
  助最富裕的国家\pagescite[][75]{davidneoliber}”。而美国可以依托美元的“世界货
  币”地位,直接在全球化中抽取高额利润。

  这同时也是民族国家与全球化中的一个课题。

\item 在放任的市场经济自由内部,本身也是金钱、金融的各方面霸权提现,而非经济自由
  和平等。如经济精英对弱势者的剥削;垄断或寡头的形成;市民税收形成的政府投入成为
  精英阶层生财资本和工具;快速私有化过程中国有、集体资本被精英阶层严重折价收购从
  而使私人资本在收购结束时就已获取超量巨额利润;金融资本远远超越生产资本占据强势
  地位,即使是凯恩斯也曾提“食利者的安乐死”,如今却是食利者的天堂。
  
\item 新自由主义确实产生了一些其它自由,如择业自由、言论自由等。但这种附带自由很
  是有限,并且
  \begin{quotation}如卡尔·波兰尼所说“这些自由在很大程度是`市场经济的副产品,这
    同一种经济也要为那些恶的自由负责'”。

    就不好的自由方面,波兰尼列出的有“剥削他人的自由,或获得超额利润而不对社会做
    出相应贡献的自由,阻止技术发明用于公益事业的自由,或发国难财的自由”。

    自由的理念由此“堕落为仅仅是对自由企业的鼓吹”,这意味着“那些其收入、闲暇和
    安全都高枕无忧的人拥有完全的自由,而人民大众仅拥有微薄的自由,尽管他们徒劳地试图
    利用自己的民主权利来获得某种保护,以免遭那些有钱人的权力的侵害”。但是——事情往往
    如此——如果“没有权力和压制的社会是不存在的,强力不发挥作用的世界也是不存在的”,
    那么维持这种自由主义乌托邦前景的唯一办法就是靠强力、暴力和独裁。在波兰尼看来,自
    由主义或新自由主义的乌托邦论调注定会为权威主义甚或十足的法西斯主义所挫。道。
    \pagescite[][38-39]{davidneoliber}
  \end{quotation}

\item 金融和跨国集团等“实力自由派”,凭借新自由主义意识形态,并非要求无所作为,
  而是试图让国家为己服务。犹如哈维在其他书中做的比喻,像一个未长大的孩子:缺奶时
  要求政府介入提供政策,为自己产奶;国家限制自身成长时,要求国家放养自己,甚至逃
  逸出国。国计民生是什么?

\item 新自由主义赖以实现的并非是“否定国家干预”。恰恰相反,\textbf{它必须大力借
    助于国家的强力、暴力和独裁,还有凯恩斯主义的赤字投入}。国家在此的作用并非是为
  了民众自由,而是被作为上层精英获利的极佳手段。

\item 新自由主义意识形态,成为最强且普遍的规训手段,为强大资本主义、现代性的理性
  代言和背书。

\item 一切成为商品,人的肉体、精神以及权利也均被作为经济商品对待,无助于实现物质
  价值的便被认为是无价值的。即使对资本精英来说,消费主义的盛行也造成表面满足、内
  心空洞、身份焦虑等。\pagescite[][179]{davidneoliber}

\item 社会、集体团结的意愿缺失,必然使人试图从他处寻得(很有限的)满足。催生黑
  社会、边缘群落、非政府组织、以及宗教团体等。

\item
  \begin{quotation}
    总体而言,新自由主义化\textbf{无法刺激经济增长或提高人民生活}。第二,从上层阶
    级角度出发,新自由主义\textbf{进程而非其理论}确实是巨大的成功:它要
    么\textbf{重建了统治精英的阶级力量}(如美国和某种程度的英国),要么\textbf{为
      资产阶级形成创造了条件}(如中国、印度、俄罗斯等等)。……简言之,不管出什么
    问题,都是因为缺乏竞争力,或因为个人、文化、政治上的缺陷。这样的论述宣称,在
    一个\textbf{社会达尔文主义}的新自由主义世界里,只有适者才应该也能够生
    存。\pagescite[][164]{davidneoliber}
  \end{quotation}

\end{enumerate}


\section{联合国债务与人权问题独立专家的历年报告摘抄}

布雷顿森林机构和发达国家常为向其它国家贷款或减债而附加\textbf{自由化、私有化、全
  球化}的条件,联合国多项机构和议题均涉及对此的强烈批判。因相关议题和文档过多,笔
者以较为随机的方式选择了\textbf{外债与人权独立专家的年度报告}作为切入点,以求管中
窥豹。

综合来看,独立专家人选不同,其倾向、水平也有不同,希望大家能够批判辩证来看。笔者
个人认为,Fantu Cheru的报告有理有据,水平极高,可作重点研究。

联合国人权高级专员办事处官网链接:\url{https://www.ohchr.org/CH/Pages/Home.aspx}。

联合国外债问题独立专家年度报告链接:\url{https://www.ohchr.org/CH/Issues/Development/IEDebt/Pages/AnnualReports.aspx}


\textbf{人权事务高级专员办事处}(联合国人权高专办)是联合国\textbf{主要的人权实
  体}。联合国大会赋予了高级专员及其办事处一个独特任务,即:促进和保护所有人的所有
权利。

\textbf{联合国人权理事会}是联合国系统中由47个成员国组成的政府间机构,负责在全球范
围内加强促进和保护人权的工作。\textbf{美国于2018年退出}人权理事会。

人权理事会是\textbf{独立于}人权高专办的实体。这种划分源于联合国大会的分别授权。尽
管如此,人权高专办为人权理事会会议提供实质性的支持,并跟进理事会所作出的评议。

\textbf{联合国外债与人权问题独立专家}\footnote{当前的全称为:国家的外债和其他有关
  国际金融义务对充分享有所有人权尤其是经济、社会和文化权利的影响问题独立专家}的职
能是就国家外债、国际金融对人权\footnote{尤其是经济、社会和文化权利})的影响问题开
展分析研究,进行国家访问任务,致力于与政府、联合国、非政府行为者和其他利益相关者
合作。


\begin{enumerate}
\item 1999年,独立专家Fantu Cheru向人权委员会提交报告,文号为E/CN.4/1999/50。报告
  认为,货币基金组织、世界银行和七国集团(简称G7)的政府官员是第三世界负债发展的
  根源。它们在债务国家未能及时还款时,一般要求债务国家以加紧实施结构调整方案(全
  球化和自由化)为条件,重订还款期限。它们的结构调整方案使债务国家陷入更为严重的
  经济和社会危机,指责货币基金组织和世界银行为“\textbf{新自由派反革命}”,“债务
  危机被用来作为打开第三世界市场,剥夺政府在国家发展中的作用的方便借口”。

\item 2000年,特别报告员Figueredo、Fanto Cheru,和独立专家Reenaldo向人权委员会提
  交报告,文号E/CN.4/2000/51。报告开篇写到:
  \begin{quotation}
    将近20年来,国际金融机构和债务国债券国政府乐于一场自欺欺人的游戏,从远距离操
    纵第三世界的经济,强行让第三世界毫无力量的国家接受不得人心的经济政策,却自认
    为宏观经济调整的苦药最终将使那些国家走上繁荣的道路和摆脱债务。
  \end{quotation}
  布雷顿森林机构,如世界银行和货币基金组织,在\textbf{全球联盟压力下}于1996年秋天
  批准了重债穷国计划(the Heavily Indebted Poor Countries,缩写为HIPC)。HIPC要求
  债务国首先“必须在世界货币基金组织的强化结构调整方案(ESAF)下完成六年的结构调
  整,然后作出减免债务的决定要满足一些额外条件”。1999年春,货币基金组织和世界银
  行在\textbf{国际大庆2000运动的政治压力下}对HIPC作检讨。
  \begin{quotation}
    \textbf{简单地说,HIPC/ESAF是国际货币基金组织和世界银行通过后门继续控制穷国和
      债务国国家发展政策的一种手段。}
  \end{quotation}

\item 2001年,独立专家Fantu Cheru向人权委员会提交报告,文号E/CN.4/2001/56。报
  告认真分析了非洲九国向IMF和世界银行提交的临时减贫战略文件(I-PRSP)。Cheru认
  可了IMF和世界银行此举积极的一面,也提出一些批评,如I-RPSP是根据捐助方设计的
  模版编制,“谈不上国家所有权的真实可靠性”,“仍导致一种把社会和人的发展以及
  公平方面的关注问题放在次于财政方面考虑因素的地位的局面”。希望IMF和世界银行
  能进一步改进。

  报告指出IMF和世界银行很大程度上服务于主要股东,即七国集团(G7)的利益,“在这
  方面,也不可忽视美国财政部的作用”。对G7不作为提出批评。

\item 2003年,独立专家Bernards Mudho向人权委员会提交报告,文号E/CN.4/2003/10。报
  告中强调了一些成功的案例,也指出
  \begin{quotation}
    \textbf{非政府组织}认为,大量证据表明结构调整战略是失败的,因为没有解决国
    际金融机构经济政策在世界各地造成的日益严重的贫困和不平等。……重债穷国倡
    议、结构调整战略和减贫扶助信贷不仅不会成功,\textbf{而且将使穷人的生活更
      加艰难},因为其目的是紧缩预算,实行的政策将缩小经济生产能力,减少可行和
    可持续性就业。

    借款人和债权人应该对重债穷国和最不发达国家日前无法持续的外债\textbf{承担共同
      责任。}
  \end{quotation}

\item 2004年,独立专家Bernards Mudho向人权委员会提交报告,文号E/CN.4/2004/47。报
  告指出,
  \begin{quotation}
    独立专家赞同世界银行业务评价部的回顾的主要结论,即《重债穷国倡议》是一项有益
    但有限的手段,必须在债务国和国际社会需要对于整体的发展筹资方法作出更广泛承诺
    的范围内加以考虑。
  \end{quotation}

\item 2005年,独立专家Bernards Mudho向人权委员会提交报告,文号E/CN.4/2005/42。
  报告延续上年宗旨,对IMF和世界银行总体持肯定态度。

\item 2006年,独立专家Bernards Mudho向人权委员会提交报告,文号E/CN.4/2006/46。
  报告中提请人权委员会注意“八国集团(G8,俄罗斯为新加入国)倡议”的进步性。
  \begin{quotation}
    由八国集团(8 国)在 2005 年夏天提出的``\textbf{多边债务减免倡议}''\textbf{预
      测}IMF、世界银行国际开发协会(开发协会)和非洲开发基金(非发基金)
    会 100\%减免世界上负债最沉重穷国的债务,以帮助这些国家实现千年发展目标

    首先,\textbf{只有成功完成重债穷国倡议的国家}(迄今只有 19 个)\textbf{才符合
      资格}。其次,只有三个多边开发银行参加债务减免倡议,使得特别是拉丁美洲和亚洲
    国家仍然承担沉重的债务。

    独立专家\textbf{遗憾}地指出,世界银行下设的国际开发协会将2003年定为取消合格债
    务的“\textbf{截止日期}”。如此改变G8的最初提议将导致债务减免大量损失。他请开
    发协会重新考虑其决定。
  \end{quotation}

\item 2007年,独立专家Bernards Mudho向人权理事会提交报告,文号A/HRC/4/10。报告中
  \begin{quotation}
    遗憾的是,近二十年期间,贸易自由化在实施上往往操之过急,并 且顺序安排不当。有
    时更易为经济教条所左右,而不是就对之可能产生的经济和社会影响作出有事实根据的
    分析。
  \end{quotation}

\item 2008年,独立专家Bernards Mudho向人权理事会提交报告,文号A/HRC/7/9。

\item 2008年,新任独立专家Cephas Lumina向联合国大会提交报告,文号A/63/289。报告
  中Cephas Lumina概述了自己的执政方针,着重指出\textbf{人权法的首要地位和人权的中
    心地位}。
  \begin{quotation}
    可以辩称,根据国际法,国家的人权义务凌驾于许多其他种类法律义务之上,因此,
    国家(以及作为国际法主题的国际组织)采取的所有行动都应与国际人权法一致。
  \end{quotation}

\item 2009年,独立专家Cephas Lumina向人权理事会提交报告,文号A/HRC/11/10。报告中
  关于债务减免所附条件的脚注是
  \begin{quotation}
    例如,根据Eurodad最近一项研究,国际货币基金每一笔低收入贷款通常附加13项条件;
    大多数条件要求实行私有化和自由论,对借款国家的穷人造成严重后
    果。
  \end{quotation}
  另一个脚注提到世界银行一份报告指出
  \begin{quotation}
    过去几十年里包括\textbf{世界银行}在内向贫穷国家提出的大多数政策忠告都强调参
    加\textbf{全球经济}的优势。\textbf{然而全球市场远非公平,其运作管理规则对发展
      中国家造成比例偏高的不利影响。}这些规则是经过复杂的谈判进程所取的结果,然而
    在这其中发展中国家却\textbf{没有什么发言权}”(加重强调)。世界银行,2006年世界
    发展报告:公平与发展(纽约:牛津大学出版社,2006年)。
  \end{quotation}
  此外
  \begin{quotation}
    过高的债务偿还以及对债务减免和新贷款的附加条件通常\textbf{限制公共开支}(甚至
    有损于向教育和保健等基本公共服务提供资金),\textbf{促进经济自由化(包括国营企
      业私有化,解除投资管理并引进公共服务使用费)},以及\textbf{优先考虑债务偿还}而
    忽视满足基本需求,这些不仅使\textbf{贫穷恶化},而且对发展中国家获得\textbf{教
      育和保健}造成尤其严重影响。
  \end{quotation}

\item 2009年,独立专家Cephas Lumina向联合国大会提交报告,文号A/64/289。报告
  中Lumina援引多方国家、组织、个人对“\textbf{非法债务}”的定义,希望依据公平、公
  正、持久、人权原则等,建立有权确立债务非法性的机构,建立负责任融资框架等。
  \begin{quotation}
    债务人和债权人共同承担责任的原则是平等的全球金融系统的核心。如同在《蒙特雷共
    识》中强调的,“债务人和债权人必须共同负责防止出现和解决债务不可持续的情
    况。”
  \end{quotation}

\item 2010年,独立专家Cephas Lumina向联合国大会提交报告,文号A/65/260、
  A/65/260/Corr.1。基金组织和世界银行对贷款和债务减免机制常常附加私有化以
  及贸易和金融部门所需的\textbf{贸易自由化}政策。报告中认为这些贸易自由化使债务
  加剧和不可持续,并对人权,尤其是经济、社会和文化权利及发展权造成不良影响和灾难
  性后果。

\item 2010年,独立专家Cephas Lumina向人权理事会提交报告,文号A/HRC/14/21。报告中
  定义了“秃鹫基金”的概念。秃鹫基金侵蚀了贫困穷国从《重债穷国倡议》和《多边债务
  减免倡议》(尽管存在种种缺陷)中获得的收益。呼吁各国立法限制秃鹫基金,其中美国、
  英国、比利时已经或者正在执行相关遏制法律。
  \begin{quotation}
    “秃鹫基金”一词用于描述私人商业实体通过购买、转让或其他交易形式,获得违约或
    不良债务,有时是实际的法庭裁决,以期获得高回报。从主权债务的角度来说,秃鹫基
    金(或如它们通常自称的“问题债务基金”)一般会在二级市场上\textbf{以远低于其面
      值的价格}获得穷国(其中很多是重债穷国)的\textbf{违约主权债务},然后企图通
    过\textbf{诉讼、扣押财产或施加政治压力}寻求获得\textbf{债务全额面值连带利息、
      罚金和法律费用}的偿付。根据非洲开发银行(非行),秃鹫基金的平均回收率为其投资
    的\textbf{3-20倍},相当于300-2000\%的利润率。非行把这种回收率描述为“可能是问
    题债务市场中\textbf{最高}的”。目前,既\textbf{没有法律限制}这种基金可通过诉
    讼获得的利息或利润总额,也没有管理框架要求披露这种基金的\textbf{购债成本}。

    秃鹫基金诉讼案件通常都在发达国家的法院上提出。这里可能是秃鹫基金的注册地或贷
    款协定中指定的管辖区。多数诉讼案件都是在\textbf{美利坚合众国、大不列颠及北爱
      尔兰联合王国和法国的法院}提起的,这些地方被视为“\textbf{有利于债权人
      的}”辖区。

    世界银行和基金组织承认商业债权人提起的诉讼“阻碍了向重债穷国提供完全的债务减
    免”。
  \end{quotation}

\item 2011年,独立专家Cephas Lumina向人权理事会提交报告,文号A/HRC/14/21。报告中
  介绍了联合国人权事务高级专员办事处,在拉丁美洲和加勒比、非洲、亚太区域,就外
  债与人权问题一般准则(“准则”)草案举办的磋商会的相关内容。

\item 2011年,独立专家Cephas Lumina向人权理事会提交报告,文号A/66/271。报告中介
  绍了出口信贷机构对国家可持续发展和人权造成的不利影响。
  \begin{quotation}
    “\textbf{出口信贷}”一词系指一种保险、担保或融资安排,它使出口资本货物和/或
    服务的购买人能推迟一段时间付款(包括通常为两年以下的短期信贷,通常为两至五年的
    中期信贷,以及通常为五年以上的长期信贷)。出口信贷是出口信贷机构提供的主要贷
    款。

    \textbf{出口信贷机构}是公共实体,向母国的私营公司提供政府担保或补贴的贷款、担
    保、信贷和保险,以支助\textbf{出口和对外投资},尤其是对发展中国家和新兴经济体
    的出口和投资。大多数发达国家至少有一个出口信贷机构,通常是其政府的官方或准官
    方机构。

    出口信贷和投资保险机构一般称为出口信贷机构。这些机构作为一个整体,是为外国企
    业参与发展中国家\textbf{大规模工业和基础设施项目}、尤其是\textbf{采掘业部门项
      目}提供公共融资的主要来源。

    出口信贷机构\textbf{以低于私人市场的利率、保险费和手续费提供融资},且这些机构
    对提供支助提出的经济条件很低,只需有限度地遵守(或根本不用遵守)环境、社会和透
    明度标准,使金融交易得以更容易、更快捷地进行,但其风险也更高。然而,对发展中
    国家的借款人而言,出口信贷机构担保的贷款的利率仍\textbf{高于}开发银行或机构等
    其他官方来源提供的许多贷款的利率。

    大多数出口信贷机构则完全没有促进发展的任务。这些机构的\textbf{唯一目标}就是促
    进本国的出口或对外投资。
  \end{quotation}

\item 2012年,独立专家Cephas Lumina向联合国大会提交报告,文号A/67/304。报告中介绍
  了国际金融机构在向借款国提供贷款、赠款和债务减免时所附加的经济改革条件,对借款
  国妇女权利造成的影响。如\textbf{削减政府开支,进行公共部门改革、公共服务私有化
    和贸易自由化}。\improve[inline]{以后做农业工业化时,可以参考这份报告。}
  \begin{quotation}
    IMF和世界银行估计,重债穷国的偿债金额占国民收入的百分比已从 2000 年的 4\%以上
    跌至2009 年的 1\%,而减贫支出占国民收入的百分比已从 2000 年的 7\%增至 2009 年
    的9\%。
  \end{quotation}
  肯定了债务减免对穷国的一些积极作用,``然而,必须强调,债务减免通常并不降低重债
  穷国的脆弱性,因为许多国家仍然严重依赖外国贷款和投资。''

\item 2013年,独立专家Cephas Lumina向人权理事会提交报告,文号A/HRC/22/42。报告中
  指出不把非法来源资金归还来源国对人权造成的影响。
  \begin{quotation}
      
    资金流动可能由两个相互重叠而又截然不同的原因成为非法流动。首先,它们可能涉
    及\textbf{犯罪所得},如腐败、贪污、贩毒或非法军火贸易。随后,所得款项常通过保
    密管辖地的\textbf{离岸存款}和为隐蔽非法资金流动专门设立的\textbf{空壳公司}进
    行清洗。其次,虽然大多数非法资金流动最初源自合法的经济活动,但是违反相关法
    律(如不支付适用的企业税或违反外汇管理条例)将这些资金\textbf{转移到国外}的做法。

    根据全球金融诚信组织的估计,2001至2010年间,下列国家外流的资金占全球非法资金
    外流总量的\textbf{76\%}:中国、墨西哥、马来西亚、沙特阿拉伯、俄罗斯联邦、菲律
    宾、尼日利亚、印度、印度尼西亚和阿拉伯联合酋长国(按非法外流资金估计量大小排
    列)。
\end{quotation}

\item 2013年,独立专家Cephas Lumina向人权理事会提交报告,文号A/HRC/23/37。
  \begin{quotation}
    虽然现行国际债务减免举措减轻了重债穷国的债务负担(从账面来看),但这些举措未能
    处理致使低收入国家的\textbf{债务无法持续}的根本原因,这些原因包括不公正的全球
    贸易条件,生产和出口基础狭窄,容易遭受外来冲击(包括国际资金量的减少),以及不
    负责任的放款等。实际上,这些举措侧重把债务降至债权人认为“\textbf{可持续}”的
    水平,这样做隐含的意思是:问题在于接受债务减免的国家在债务管理上欠谨慎而且治
    理不善。\textbf{债权人在举措中发挥主导作用,这是与共同责任原则相抵触的。}同时,
    与举措相关的附加条件损害了债务国的主权,在某些情形中妨碍了债务减免的减贫目标
    的实现。
  \end{quotation}

\item 2013年,独立专家Cephas Lumina向联合国大会提交报告,文号A/68/542。报告主要
  根据《联合国千年发展目标差距工作队2012年报告》,总结千年发展目标8与实际进展的
  差距,希望建立更为强大的发展框架。
  \begin{quotation}
     2000 年举行的联合国千年首脑会议上,各会员国“决心在国家一级和全球一级创造一
     种有助于发展和消除贫穷的环境”(见第55/2号决议,第 12 段)。关于建立全球发展
     伙伴关系的千年发展目标 8 随后阐述了这一承诺。目标 8 包含了增加官方发展援助、
     最贫穷国家的市场准入、减免债务、获得基本药物、技术转移等领域的一些具体承诺,
     而且应特别考虑到最不发达国家、内陆发展中国家和小岛屿发展中国家的需求。
   \end{quotation}
   笔者摘录报告中针对千年目标8实际进展过程中的个别不足和缺陷如下:
   \begin{quotation}
     发达国家持续的农业补贴也继续对发展中国家的农业贸易和生产产生不利影响。2011
     年,经合组织国家的农业补贴增加到国内生产总值的 0.95\%。虽然发达国家农业补贴
     一天为 10 亿美元,许多贫穷发展中国家无力补贴其农业,导致其产品价格较高,农
     民的贫困加剧和生活水平下降。发达国家还对进口的制成品和加工产品征收高额税,
     使得发展中国家无法赚取更多收入,并使他们只限于原材料出口。贸易谈判多哈回合
     停滞不前使得问题进一步复杂化。

     二十国承诺抵制所有保护主义措施,并纠正任何在应对全球金融危机中所采取的保护主
     义措施。但2008世界贸易自危机开始以来只废除了一小部分已经实施的贸易限制。迄今
     实施的贸易限制已影响到将近 3\%的世界贸易。遭到\textbf{保护主义措施}的影响。

     如独立专家在提交人权理事会的报告(A/HRC/23/37)中所指出,\textbf{债务减免的直
       接财政影响难以衡量,债务减免与减贫支出增加之间的因果关系也难以确定}。

     债务减免机制已经\textbf{完全为债权人所主导},过度侧重于纠正被视为是受援国方
     面不慎重的债务管理,没有债务解决问题的根本原因,包括不公平的贸易条件、不负责
     任的贷款和国际金融机构不当的政策规定。

     目标 8 下的其他具体目标包括加强在发展中国家提供负担得起的基本药物和新技
     术(特 别是信息和通信技术)。增加提供负担得起的基本药物对实现与卫生有关的千年
     发展目 标和实现健康权十分重要。……但是,根据跟踪实现千年发展目标进展情况的
     报告,近年来发展中国家在改善基本药物的提供和负担能力方面进展甚微。根据《千年
     发展目标差距工作队2012 年报告》,2007-2011 年期间平均只有 51.8\%的公共保健设
     施和 68.5\%的私营保健设施提供基本药物。
   \end{quotation}

\item 2013年,独立专家Cephas Lumina向人权理事会提交报告,文号A/HRC/25/52。报告主要
  提及不把非法资金归还来源国对享受人权的负面影响。
  \begin{quotation}
    “\textbf{非法资金}”一词泛指腐败、贿赂、贪污、逃税和其他犯罪行为的收益。
  \end{quotation}
  
\item 2014年,独立专家Cephas Lumina向人权理事会提交报告,文号A/HRC/25/50。报告中
  Lumina总结了自己2008-2014任期内的工作。
  \begin{quotation}
    人权理事会的一些成员没有为(外债和人权)这一任务提供支助,尤其是欧洲联盟和美
    利坚合众国。(笔者注:在人权委员会就结构调整和经济改革政策及外债对充分享有所
    有人权尤其是经济、社会和文化权利的影响问题的决议和决定中,\textbf{中国}几乎一
    直投赞成票,英法意德韩日等常投反对票)
  \end{quotation}

\item 2014年,独立专家Juan Pablo Bohoslavsky向联合国大会提交报告,文号A/69/273。
  报告中说明了Bohoslavsky在2014 - 2017 年期间的初步工作计划。
  
\item 2015年,独立专家Juan Pablo Bohoslavsky向人权理事会提交报告,文
  号A/HRC/28/60。报告主要涉及非法资金流动问题。
  \begin{quotation}
    根据全球金融诚信组织的最新\textbf{估计},发展中国家在2012年因非法资金外流损失
    了\textbf{9912亿美元},比2011年再增加1.8\%。2003年以来,非法资金外流实际每年
    增加9.4\%。通过将这些数字与发展中国家收到的官方发展援助比较即可表明这种资源流
    失的规模。2012年的官方发展援助为897亿美元,这意味着,2012年每支出\textbf{1美
      元}的发展援助,就有超过\textbf{10美元}以\textbf{非法资金外流}的形式逃离发展
    中国家。根据全球金融诚信组织的资料,过去十年官方发展援助和外国直接投资加在一
    起也抵不上发展中国家的非法资金外流。

    根据保护记者委员会的资料,截至2014年12月31日,在1992年以来全球被谋杀的725记者
    中,有208人或29\%报道了\textbf{腐败问题}。记者无国界组织2011年报告,
    在2000-2010的十年中至少有141名报道\textbf{有组织犯罪和贩毒}——非法资金流动的另
    一个主要来源——的记者被杀害。
  \end{quotation}

\item 独立专家Juan Pablo Bohoslavsky向人权理事会提交报告,文号A/HRC/28/59。报
  告“主要关注金融共谋:向严重侵犯人权的国家提供贷款问题的报告”,提议对这些国家
  的贷款必须经过评估和验证。
  \begin{quotation}
    向严重侵犯人权的政权提供贷款可能有助于政权巩固、使不尊重人权行为得以延续以及
    增加严重侵犯人权的可能性。这些结论对官方和私人对政府的金融援助都适用。然而,
    私人贷款似乎更具有破坏性,因为与国家间的贷款和由国际金融机构分配的贷款相比,
    私人贷款的公共问责程度较低。
  \end{quotation}
\end{enumerate}

% https://wallstreetcn.com/articles/230968
% https://en.wikipedia.org/wiki/Vulture_fund
% https://en.wikipedia.org/wiki/Vulture_capitalist
% http://www.xinhuanet.com/fortune/2016-03/01/c_1118200280.htm
% https://botanwang.com/articles/201407/%E7%9C%8B%E7%BE%8E%E5%9B%BD%E7%A7%83%E9%B9%AB%E5%9F%BA%E9%87%91%E5%90%91%E9%98%BF%E6%A0%B9%E5%BB%B7%E9%80%BC%E5%80%BA%E7%9A%84%E6%89%8B%E6%B3%95.html
% https://wiki.mbalib.com/wiki/%E7%A7%83%E9%B9%AB%E5%9F%BA%E9%87%91
% http://www.argchina.com/wx-index-content-id-3325.html

% http://cshan.mrecic.gov.ar/zh-hant/content/%E8%81%94%E5%90%88%E5%9B%BD%E5%A4%A7%E4%BC%9A%E6%89%B9%E5%87%86%E9%99%90%E5%88%B6%E7%A7%83%E9%B9%AB%E5%9F%BA%E9%87%91%E8%BF%90%E4%BD%9C%E5%87%86%E5%88%99

% https://news.un.org/zh/story/2010/04/129822

% http://genevese.mofcom.gov.cn/article/sqfb/201603/20160301270885.shtml

% https://documents-dds-ny.un.org/doc/UNDOC/GEN/G10/131/55/PDF/G1013155.pdf?OpenElement

% https://www.ohchr.org/CH/Issues/Development/IEDebt/Pages/Debtrestructuringvulturefundsandhumanrights.aspx
% https://wallstreetcn.com/articles/230968
% https://en.wikipedia.org/wiki/Vulture_fund
% https://en.wikipedia.org/wiki/Vulture_capitalist
% http://www.xinhuanet.com/fortune/2016-03/01/c_1118200280.htm
% https://botanwang.com/articles/201407/%E7%9C%8B%E7%BE%8E%E5%9B%BD%E7%A7%83%E9%B9%AB%E5%9F%BA%E9%87%91%E5%90%91%E9%98%BF%E6%A0%B9%E5%BB%B7%E9%80%BC%E5%80%BA%E7%9A%84%E6%89%8B%E6%B3%95.html
% https://wiki.mbalib.com/wiki/%E7%A7%83%E9%B9%AB%E5%9F%BA%E9%87%91
% http://www.argchina.com/wx-index-content-id-3325.html

% http://cshan.mrecic.gov.ar/zh-hant/content/%E8%81%94%E5%90%88%E5%9B%BD%E5%A4%A7%E4%BC%9A%E6%89%B9%E5%87%86%E9%99%90%E5%88%B6%E7%A7%83%E9%B9%AB%E5%9F%BA%E9%87%91%E8%BF%90%E4%BD%9C%E5%87%86%E5%88%99

% https://news.un.org/zh/story/2010/04/129822

% http://genevese.mofcom.gov.cn/article/sqfb/201603/20160301270885.shtml

% https://documents-dds-ny.un.org/doc/UNDOC/GEN/G10/131/55/PDF/G1013155.pdf?OpenElement

% https://www.ohchr.org/CH/Issues/Development/IEDebt/Pages/Debtrestructuringvulturefundsandhumanrights.aspx



%%% Local Variables:
%%% mode: latex
%%% TeX-master: "../main"
%%% End:
