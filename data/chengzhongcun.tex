\chapter{城中村的简单概念}

所谓城中村者,城市和村庄性质兼而有之:它深处城市之中,作为城市的一部分,周边均具
城市特征,它自身却充斥着农村式的无序和自然,缺乏人工的总体规划,各家各户的宅地界
限比纯粹的村庄还要不清晰和混乱,基础设施(能源、通讯、供水、交通、安全、卫生、医
疗、文化等)薄弱,原生住户基本为农村户籍,土地制度仍为农村集体所有制而非城市的全
民所有制;作为农村,它的外来流动人口数量数倍,甚至数十倍于原生居民,耕地被大量或
完全占用,转为商业或住宅地产,耕地的这种性质转变使原生居民原先赖以生存的农业收入
转为地产收入,并成为原生居民收入的重要来源。

2016年《联合国人居三筹备委员会第三届会议政策文件10:住房政策》(文号A
/CONF.226/PC.3/23)对贫民窟词条所作的解释为:
\begin{quotation}
  人居署《世界城市状况》文件指出,贫民窟的生活和环境条件最为恶劣,诸如供水不足,
  卫生恶劣,住房拥挤且破旧不堪,所处地点存在危害,保有权无保障,易受严重健康风险
  的危害。

  自2003年起,联合国会员国商定将\textbf{贫民窟家庭}定义为生活在同一屋檐下,但缺乏
  以下五项条件中的一项或多项的一组个人:(a) 能得到经改善的饮水;(b) 用得上经改进
  的卫生设施;(c) 充足的居住面积,不过于拥挤;(d)住宅的结构质量/持久性;(e) 土地保
  有权的保障。
\end{quotation}

结合中国政府棚改文件,我们通常所说的城中村属于中国政府定义的“棚户区”,且多
为“\textbf{城市棚户区}”,城中村也符合联合国所定义的贫民窟。\cite{unandchina}


中国的城市贫民窟人口有多少呢?联合国人居署根据它对贫民窟的定义测算数据如下:
% Please add the following required packages to your document preamble:
% \usepackage{booktabs}
\begin{table}[!ht]
\centering
\caption{1990-2014年中国城市贫民窟人口比例及数量}
\capsource{联合国人居署旗舰报告《World Cites Report 2016》\cite{9789211327083}}
\label{my-label}
\resizebox{\linewidth}{!}{%
\begin{tabular}{@{}lllllll@{}}
\toprule
                & 1990年  & 1995年  & 2000年  & 2005年  & 2010年  & 2014年  \\ \midrule
城市中贫民窟人口的比例(\%) & 43.6   & 40.5   & 37.3   & 32.9   & 29.1   & 25.2   \\
城市贫民窟人口的数量     & 1.316亿 & 1.514亿 & 1.691亿 & 1,835亿 & 1.806亿 & 1.911亿 \\ \bottomrule
\end{tabular}%
}
\end{table}

根据《国家新型城镇化规划(2014-2020年)》,我国预计“到2020年基本完成城市棚户区改
造任务”。根据李克强总理在2018年第十三届全国人大一次会议所作政府报
告\footnote{\url{http://www.gov.cn/gongbao/content/2018/content_5286356.htm}},“棚户区
住房改造2600多万套,农村危房改造1700多万户,上亿人喜迁新居”。全国轰轰烈烈兴起的
城中村改造跑步前进,新型城镇化取得了惊人成绩,这同时也标志着,曾经遍布每个城市的
老式城中村的大量消亡。

\improve{之后将下一章改为 cleveref,指明具体章节。}下一章笔者介绍自己对济南市丁家
新村的简单调查和感受,下下章笔者将结合各方文献总体探讨中国式空间生产并提出批评和
建议。


\chapter{济南市丁家庄城中村社会调查报告(未完成)}

\section{背景介绍}

济南市丁家庄,又名丁家村、丁家新村,据1992年5月1日所立村碑记载:
\begin{quotation}明永乐年间(1403-1424)当地根据传说取村名“定妖庄”。后因此名不
雅,故以“定”字谐音“丁”字改为丁家庄。
\end{quotation}

丁家庄隶属于山东省济南市姚家街道,是济南市一个较大且密集的城中村,在二十年前就已
开始为外来务工人员提供住房等服务,共有村民宅基地(院落)近800户5000人,外来流动
人口峰值大约可达30000人。丁家村城中改造是山东省棚改旧改的重点项目,于2017年年底
基本完成房屋拆除工作,拆迁面积约为53万平方米(村民宅基地、公益性公共设施用地加上
经营性用地等)。

\section{丁家庄社会调查起因}

2014年,笔者的工作和住宿地点均变更为济南市化纤厂路,距离主要供应农副产品和小规模
餐饮服务的丁家庄综合市场不足500米。2016年7月份,笔者为练习街头摄影常在街头漫无目
的地游荡,有次穿过菜市场发现了一片新景象——丁家庄城中村。当时笔者虽已在济南工作9
年,去过济南市诸多地方并在丁家庄附近生活2年,但从未注意过有丁家庄城中村这样一个存
在。

\improve{引入奥体西路丁家庄城中村改造安置房项目 规划以及建设情况}

笔者本人就出生和成长在一个贫困县,也去过几次农村,但初入丁家庄时仍因它表面的破败
和杂乱而产生一种恐惧感,主要是对治安的恐惧(像许多社会田野调查一样,这种先入为主
的治安恐惧经过假以时日的了解之后被证明绝大部分多余)。这里六层建筑甚少,后经询问
整个丁家庄只有20余栋拥有三、四单元的六层楼,其余多为村民自建并层层加盖的三四层楼
房,个别楼房房顶搭有活动板房。道路多为水泥石板路,蜿蜒曲折缺乏整体规划,笔者游历
丁家庄数次之后才可不迷路。粗细不一、聚成一团的电力、通讯线缆杂乱丛生,穿越整个村
落。\improve{缺乏对居住人口的描写}

虽如此,何不拿这里作为一个摄影的练习场所呢?拍有意义的照片而非漂亮的照片吧,但这
一想法始终被困在如何发掘和表现丁家庄城中村的意义上。后来网友深圳的邱文提出了宝贵
的建议——以人类学的视角来直接记录丁家庄的衣食住行和生活在里面的各色人群,作为即将
消逝的中国城中村的一个存档。在邱文建议的基础上,笔者查找资料和思考后决定以社会学
调查的方式来展现丁家庄城中村。

\section{失败的丁家庄城中村社会学调查}

笔者起初准备以定量研究方法结合定性研究方法对丁家庄作一个较为全面的社会调查。其中
定量方面,仿照ISSP \footnote{The International Social Survey Programme,国际社会
调查方案}和中国的CGSS \footnote{Chinese General Social Survey,中国综合社会调查,
于2007年加入ISSP。}问卷作一个针对丁家庄城中村的调查问卷,定性研究初步决定采用
Phil Francis Carspecken的批判定性研究框架。

之后笔者又十几次进入丁家庄,也曾在丁家庄居住做过1个月进行田野调查,但因自身怠惰和
三心二意使本次调查大失败,不过以下几点心得或许有益于类似社会调查的开展,在此分享
给读者:
\begin{enumerate}
 
\item 定性研究一般要求记录谈话,录音常是记录谈话的主要方式。扎根理论和Carspecken
的定性框架必须建立在大量谈话资料的多次整理上。但丁家庄人均谈录音色变,拒绝录音。
这种拒绝主因是被调查人在敏感性的事件上害怕录音成为某种对自己不利的证据。

\item 城中村人员组成和住房结构的复杂,使针对整体的概率抽样问卷调查非常困难。实际
上笔者认为,只有具有政府背景的组织或机构大力支持、推动才能完成类似复杂区域定量研
究的概率抽样。

\item 笔者采用了非概率的随机街头访问方式发放调查问卷,这使调查结果可能产生无效的、
完全不具代表性的样本。并且即使如此,问卷回收率仍然极低,只能勉强算是10\%,使定量
分析成为不可能。

\item 利益敏感问题信度不高。除笔者本人能力拙劣外,也有现实客观。例如针对房东的调
查中,房东往往隐瞒和减少实际出租房屋间数以及出租收入,可从被调查房东所处房子建筑
外观、体积以及丁家庄出租房的平均面积和收入得出这一信度不高的结论;针对所有人的收
入问题也存信度问题,再三询问或试探所得出的收入结论最高浮动为1000多元人民币。调查
问卷中针对村委和拆迁方案的满意程度采用了5级李克特量表,但被调查人极端选择较多,
情绪化明显,个人利益最大化主导的倾向明显。

\item 半数以上房东有对上级政府机构的强烈诉求,这也是他们对社会调查人最大的期望。
  本次调查为笔者个人自发,没有任何组织机构背景,也一再向被调查人言明本人所写报告
  预计不会产生任何一点社会影响力,无法满足房东这种诉求。除此之外,社会学可以采取
  小额金钱奖励的方式来增加被调查人积极性,但笔者着实囊中羞涩,无法采用这种方式。

\item 租房人对本次社会调查表现出严重的整体冷漠,可以认为这是一种社会排斥。关于这
  方面内容,笔者放在之后章节再详细论述。小额金钱奖励应可以有效提高租房人积极性,
  但未实施,原因同上。

\item 最主要原因仍是笔者个人社会调查能力的欠缺,和态度的不端正。笔者接触社会学是
在丁家庄摄影项目受阻之后从零开始,在社会学意义上的与人交往也存各种缺陷。最主要的
还是态度,三心二意、半途而废,甚至因屡次消沉而遗失了几份已经回收的完整调查问卷。

\end{enumerate}

本次调查过程中,笔者听闻有两位女士几乎同期在丁家庄城中村进行社会问卷调查,并对问
卷完成者提供每人50元奖励,效果不错。笔者估计是具有政府背景的组织机构,如大学在做
这份工作。希望我国能够在当前基础上进一步普及社会学调查相关知识,增加社会学调查项
目,并保证社会学调查的中立性及公信度,同时也希望调查者能够坚守信度和效度问题。

下一节笔者将介绍在丁家庄中的所见所闻所感。

\section{丁家庄见闻散记}

\subsection{丁家庄城中村的黑夜与清晨}

丁家庄城中村的夜总比周边来的更早些。

街道路灯不多,住户家中多使用散发着黄色光晕的白炽灯,晚上八九点钟,黄色的主色调混
合着个别店铺的七彩霓虹灯光闪耀在城中村里,叮叮当当的做饭声时常响起,在笔者居住的
楼房,笔者还见过一楼一个不足10岁的小男孩在独立地炒菜做饭。深夜,村外尚有较晚收摊
的小吃车、大排档、夏天24小时营业的烧烤店、配合餐饮的小零售店、长明的路灯和过往的
车辆,村里却是另一片景象,这里更黑更静。晚10点左右还能偶尔听见晚归人家的锅碗瓢盆
交响曲,11点多整个村子就一下子寂静起来,水泥石板路上鲜有路人。笔者有次晚归路过一
个没有路灯照耀的环卫点,看见一位老人在几个垃圾箱中翻找可再利用或可卖的杂物,黑夜
掩护了他的自尊。

在丁家庄外,有家24小时营业的包子铺,据说那间包子铺在凌晨的主要顾客是性工作者和他
们的客人。许是前几年经历过严打,笔者并未发现丁家庄里有较为明显的提供性服务的店家。

这里的早晨也比周边来的更早些,6、7点钟雄鸡打鸣,各家各样的声音均透过不隔音的墙壁
和窗户传到家家户户。孩子、送孩子上学的家长、上班族一下子散布在各处,城中村在这个
时间已经开始繁忙起来。

\subsection{医疗保健和社会保障}

笔者有次走出丁家庄时,碰见村口一位50岁左右的男子坐在轮椅上斜着头,面无表情、眼神
空洞,对周围不管不问地在晒太阳,或许是偏瘫。未过几秒,迎面又走来一个怯生生的30岁
左右的男子。他提着午饭低头走来,看见我时便将整个身子直接旋转180度,用后背对着我,
不敢和我有一瞥眼的接触。当我正要和他擦肩而过时,这位男子又朝无人那侧180度急速转身,
继续前行。笔者并不魁梧有力,对他人几无压迫感,这位30岁左右的男子应有视线恐惧症或
社交恐惧症等精神疾病吧。

因笔者能力有限、怠惰和调查时间选择上的问题,被调查人群多为中老年人并且数量很少。
虽然存在这种样本偏差导致不能推导出一般性的结论,但笔者认为可以提下自己直观和经验
感受:不管是流动人口还是常住人口,他们身高较之周边明显偏矮,心脑疾病较多,也有被
调查人家庭两代人中均患重大疾病的事例。

《中国心血管病报告2017》中开篇有提“我国居民心血管病(CVD)危险因素普遍暴露,呈现在
低龄化、低收入群体中快速增长及个体聚集趋势。”,但在“CVD危险因素”一节所列举的九
个致病因素中,并无贫穷,几乎全是病理性因素。笔者认为融入社会学视角,考察人因受过
贫穷或仍处贫穷状态所身处的生活环境和个人习惯对自身健康带来的恶劣影响很有必要,贫
穷也是一种病、顽疾,甚至是绝症,期待国家、组织、机构、个人能做这方面的深入调查。

根据“丁家庄环境卫生管理公示牌”,丁家庄有保洁人员21人,保洁面积4万平方米。据本次
调查,济南市环保局贯彻执行八小时工作制,并为保洁人员缴纳三险。所聘用保洁人员多为
丁家庄居住人口,每月到手收入在1600元左右。济南市环保局在劳保上的表现出乎笔者意料,
在此点赞。另外,有一例保洁员工伤纠纷,当事人为外地来济60多岁老人,因是否算工伤与
环保局有分歧,环保局领导表示亲切慰问但未解决分歧,言谈中笔者感觉当事人并不会积极
争取工伤赔偿。即使如此,当事人对国家和政府仍表示“非常满意”(满意度为五个量
级,“非常满意”是最高级。),生活美满,没有任何意见。

丁家庄老年村民个人大多没有缴纳任何形式的养老保险,包括新型农村社会养老保险(新农
保),他们也不清楚新农保的具体政策。满60岁老人由村委每月补助600元左右(TODO:笔者
尚不清楚这份补助是不是由村委为村民代缴新农保而产生的新农保支出;也不清楚拆迁后这
份补助是否仍然继续发放)。丁家庄大部份村民所能缴纳的社保只有新型农村合作医疗(新
农合),且只有每年缴纳100元和300元两个档次。新农合在丁家庄村民重大疾病治疗上发挥
了极其重要和显著的作用,常可报销60\%多的费用。

\subsection{他人即地狱}

在丁家庄居住的一个月里,笔者常去丁家庄外围一家豆腐脑店吃早点。2017年4月中旬的一天
清晨,笔者如同往常一样来到这里。店里帮工的一位中年妇女反应迟钝,记性不佳,行动迟
缓,笔者在数分钟时间里反复说明我想要的早点,可这位中年妇女似乎一次也没听进去。眼
看着快到上班时间了,笔者带着不耐烦和火气,嚷道“老板,我要的XXX什么时候能上?都要
了好几遍了”!

这位中年妇女大约是前两天来此店铺帮工,也曾有几位客人不耐烦地指责她业务不熟练,可
她为自己辩解的的理由总是一步步将自己推向更深的深渊,例如不急不躁、慢条斯理的说
道“我就是过来临时干干,过两天就走了,不用熟练啦”……

笔者也见过这种人,他们往往从幼时就要经历家庭的困苦和生活的波折,另一方面又执念于
获得自尊。客观条件、个人某方面能力的缺失(其他方面可能具有超常人的能力)使他们无
力获得众人对自己的认同,也使他们保护自尊的方式拙劣,这种拙劣往往又反过来更加伤及
其本欲保护的自尊,犹如溺水之人的挣扎于事无补,只能使自己越陷越深,在挣扎过程中,
众人的打压又使其溺水程度进一步加深。他人往往成为这种人的地狱,这里的他人,甚至可
以包括家长、夫妻或子女。

发完火, 笔者忽感惭愧:我明明知道这些,我怎么还发火了?我岂不是变成了自己曾经讨厌
的那种人?笔者可以为自己辩解:我花了钱,所以我要享受相应的服务;中年妇女服务不好,
我们可以指责他。但这真的那么正确吗?具体的、特殊的人在这种想当然正确的道德标准中
被置于何处呢?笔者主观感觉这里面存在明显问题却碍于自身所知受限无法给出一个答案,
留待读者探讨吧。笔者隔天再去这家店时,这位溺水妇女已经不在了……

\subsection{何不食肉糜}

虽然丁家庄北侧就是一个较大的综合市场供周边几个社区居民采购农副产品,但城中村里仍
有些蔬果摊子长期固定在街道一角,它们铺设在水泥道路或机动三轮车上;还有些定期定式
流动叫卖的轻型贩菜货车。城中村蔬果摊主要面向城中村内中老年居民,人流量和购买力远
不如综合市场,与综合市场高额租金相对比的是城中村菜摊无需承担任何地租,多供应次一
级的菜品并且价格低廉。 (可参考\cref{fig:caishichang}, \cref{fig:caitandajie1},
  \cref{fig:caitandajie2}。)

这里我们来谈一个架设在机动三轮车上的水果摊主吧,为方便起见,笔者将其隐代为“果
王”。笔者在田野调查中,据几个丁家庄外来人介绍,果王可能是居住环境最为糟糕的,他
一家三代住在一个200多元月租的单间中。虽然笔者动过拍摄他家居环境的想法,也向果王透
露过这个想法,但最终还是决定不可再提。无论丁家庄社会调查的成效如何,这样的拍摄对
于果王一家人都将是一种赤裸裸地侵犯,并且这种侵犯对其家庭成员、对像果王这类穷苦人
来说,几乎绝不会产生任何有益改变。这种照片的运用,只能让调查显得“真实鲜活生动”,
实则不能摆脱上位者“消费苦难”的事实,文字足够了。

果王的奋斗史,同样也是“落后”小贩被步步驱逐的历史。他做过走街串巷流动叫卖的小贩,
被驱逐淘汰;又做过小区外较固定的摊贩,被驱逐淘汰;又做过丁家庄外路边的摊贩,同样
也是被驱逐淘汰;最后果王成为了一个丁家庄内机动三轮车上的小贩。他的朋友向我介绍果
王这些历史时情绪激动,而果王也泪湿双眼。最后,丁家庄已被夷为平地,果王不知又将以
何种方式去继续书写他自己的奋斗史……

夏季的一天,笔者碰到果王十岁左右的儿子从自家中捧来几片薄切的西瓜给果王吃。可他明
明就在卖瓜呀。

笔者无论如何也没有想到,当笔者向一些人说起丁家庄综合市场和城中村菜摊的区别时,有
些人会指责果王不努力不争气,质疑果王为何不早在综合市场租摊位(高租金)以求得良好
收益。一个一家三代居住在城中村破败单间的外来人,如何去承担每年数万的租金啊。简单
用资本作为人的衡量标准,从而一叶障目时,我们都将成为那个呆傻可笑的晋惠帝,“百姓
无粟米充饥,何不食肉糜?”

\subsection{过往的宅田基地之殇}
% “社会空间” 概念。他把社会空间理解为 “空间的实践”、 “空间的表象” 和 “具象的
% 空间”的三位一体。也就是说,空间不是一个抽象的存在,而是自然的、精神的和社会的三位
% 一体,这个一元的整体性的社会空间既是逻辑-认识论的空间,同时也是社会实践的空间和亲身
% 经历包括感觉想象的空间。通过这个包含着诸多规定和关系的社会空间概念, “社会空间”
% 概念。他把社会空间理解为 “空间的实践”、 “空间的表象” 和 “具象的空间”的三位
% 一体。也就是说,空间不是一个抽象的存在,而是自然的、精神的和社会的三位一体,这个一元
% 的整体性的社会空间既是逻辑-认识论的空间,同时也是社会实践的空间和亲身经历包括感觉
% 想象的空间。通过这个包含着诸多规定和关系的社会空间概念,

% 列斐伏尔强调,三元分析的三个方面是既相互独立又相互作用的,三重要素中每一个方面都可
% 以也应该在与其他两个方面中的任意一个的相互关系中得到理解。并且任何一个要素对于其
% 他两个都不具有绝对的优先地位。

% 社会空间是 “空间的实践” (spatial practice)、 “空间的表象”(representations of
% space)和 “具象的空间” (representational spaces)的三位一体。社会空间的生产可以从
% 这三重概念的辩证运动中得到理解。相应的,如果我们从身体出发理解社会空间,这三重图式
% 又可以表达为 “感知的” (the perceived)、“概念的”(the conceived)和 “经验
% 的”(the lived)。


% “空间的实践” 包括生产和再生产,以及构成每一社会的典型的具体场所和空间化
% 的位置。一个社会的空间的实践通过保证连续性和凝聚力,生产并隐藏这一社会的空
% 间。

% “空间的表象” 是概念化的空间。

% “具象的空间” 既是作为 “居住者”、“使用者” 或 “占用者” 的人们生活于其中的空
% 间,包括房屋、教堂、广场等;也是一些艺术家、哲学家、作家们所描述但又渴望不仅
% 仅是描述的空间。因此,一方面,具象的空间与人的真实的生活经验相连,与空间的表

宅、田基地矛盾曾是中国小农经济为主的农村中典型、多发的矛盾,且这一矛盾的表现往往
采取激烈、甚至极端的方式。在笔者对丁家庄的走访过程中,也曾碰到宅基地纠纷当事人商
姐说起一例二十多年前的惨剧:两家因新修墙壁越界而产生的宅基地矛盾步步升级,致使事
件一方家主自杀,另一方家主被群殴打个半死,双方之间的经济、土地纠纷至今未解,被搁
置一旁二十余年。商姐诉说到情绪最为激动时,那种神情让笔者害怕想逃,从未见过这种强
度的艰辛、怨愤、痛苦和不甘,彼此交织在一起……

在中国传统小农经济的农村中,往往以家庭为基本单位,主要活动范围局限在村内,生产集
中在自家规模极为有限的耕田或从事简单手工业、半加工业的家中,生活集中在自家住宅与
村内公共空间。田区和住宅区常常分隔明显,呈大块状分布,同一大块内常是多家彼此有联
结的田地或住宅,其中相联两家之间的田地多用沟、垄、界石作为田界,住宅多用共用的一
面墙壁或距离极近的两面墙壁作为宅界。

小农经济中,多数人的社会空间长期固定、聚合、封闭在本村,物质和社会资源有限,生产
生活单调贫乏,村民之间联系频繁,信息传播速度快,这种情况下发生的利益冲突常尖锐和
持久,田界和宅界作为重要家庭产权的界限尤其不容侵犯,具有相当刚性。

在实践和具体的社会空间中,这一刚性界限却又常常变动并受到侵蚀。它本身包含农村中通
风、采光、日照、排水、通道等难以界定的方面,另外在国家和政府层面来说,又有历史遗
留、立法不健全、执法成本高等问题;在村民来说,则有历史遗留、违规超额占用现象普遍、
法律维权成本高、法制观念不强、宗族势力等问题。如果此家庭被其他家庭侵犯界限而未采
取有效措施,则不单是家庭经济效益,连带个人的自我认同、社会地位、在家庭中的地位以
及家庭在村内的地位也将受到严重的负面影响。这也使宅、田基地矛盾相当尖锐频发,家庭
中的强壮男子往往被赋予保卫这一刚性界限的责任。

在90年代中期,丁家庄的耕地开始大量转承给企业,村民也开始租屋建设,实际使用的宅基
地大规模外扩并层层加高,外来人口流入,青年一代外迁,资本观念逐步加强。正是生产方
式的改变使村庄从原本的封闭转为面向所有人和整个社会,并超越其他方面成为社会衡量标
准,一些落后观念被弱化或淘汰。宅基地矛盾在城中村经济中应不如小农经济时那样典型、
激烈,但它的矛盾点未曾改变,只能被弱化不可能被大规模消除。

丈夫自杀后,商姐带着两个儿子独立生活,其中小儿子当年不到两岁。她抓住了空间转化的
时机,是90年代丁家庄第一批建立租屋的人。丁家庄拆迁前她的租屋总面积已经超过1500余
平,出租房屋超过50户,月入过万,即使在丁家庄这也是了不起的成就。商姐后来又重新组
建了家庭,并抚育两个儿子都考上大学。长年的劳苦使其腿病明显,面态老相,但商姐勤劳
不改,晚上仍会去丁家庄综合市场进行清场打扫工作。

商姐丈夫自杀后,商姐及其亲朋殴打了另一方家主,并拒不履行法院关于此次斗殴生效判决
中的经济赔偿责任,另一方家主以此为理由数年来阻止商姐开发属于她家的一块空地,这块
空地已经自然形成一块公共停车场地。当商姐以维权为理由向我倾诉她的故事时,笔者一再
说明这个小项目不能给其带来任何改变,也难以根据她的一家之言去支持她,她对笔者的中
立表示认可,最后甚至还埋怨起她的家庭当年为什么不让那寸土的界限,即使再让更多些也
不要紧啊……诉说过程中,笔者渐感商姐并非真要维权,并非还存那样的恨意,她真正想要
的其实只不过是一场没有利害关系的倾诉。

为使商姐的诉求不至于完全落空,笔者提出以中立态度拍摄那块未开发空地的想法,商姐当
时对此表示支持。但是之后笔者几次联系商姐时,她均明显回避,这也证明了倾诉才是她的
主要目的,而非仇恨:倾诉完之后当事人产生了羞愧感,不敢面对被倾诉人;而所谓仇恨早
已经在时空的变换中变成了一介难以抹去的疮疤。随着丁家庄旧村改造的接近完成,丁家庄
人将进入新的、彻底的城市化空间,城市化空间中刚性界限更为明显也易维权,原本农村中
典型的宅基地矛盾将不复存在,希望两家日后能够相忘于江湖,也同样希望人们彼此能够多
给一些倾诉的空间以使悲剧不再那么难以承受。


% 丁家庄城中村是一个充斥着盎然生机、孕育着诸多可能的城中村。它是许多人实现梦想的起
% 步点或中转站,也展开怀抱接纳了各方边缘人群,诸多住户之间较为和谐。它也远未成为一
% 个堕落之处,这里并非治安恶化严重,

%%% Local Variables:
%%% mode: latex
%%% TeX-master: "../main"
%%% End:
