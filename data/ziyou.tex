\chapter{新自由主义}

这个领域确实是天赋人权的真正伊甸园。那里占统治地位的只是自由、平等、所有权和边沁
\footnote{杰里米·边沁(Jeremy Bentham,公元1748年2月15日—公元1832年6月6日)是英
  国的法理学家、功利主义哲学家、经济学家和社会改革者。有用哲学即功利主义的代表人
  物之一。对他来说,个人的利益是一切行动的动力。然而,一切利益,如果正确加以理解,
  又处于内在的和谐状态中。各个人的正确理解的利益也就是社会的利益。}。自由!因为
商品例如劳动力的买者和卖者,只取决于自己的自由意志。他们是作为自由的、在法律上平
等的人缔结契约的。契约是他们的意志借以得到共同的法律表现的最后结果。平等!因为他
们彼此只是作为商品占有者发生关系,用等价物交换等价物。所有权!因为每一个人都只支
配自己的东西。边沁!因为双方都只顾自己。使他们连在一起并发生关系的唯一力量,是他
们的利己心,是他们的特殊利益,是他们的私人利益。正因为人人只顾自己,谁也不管别人,
所以大家都是在事物的前定和谐下,或者说,在全能的神的保佑下,完成着互惠互利、共同
有益、全体有利的事业。