\chapter{个人的社会学}

我正在做一个个人自发的电子书项目。这本电子书将是个四不像但又什么都像的怪物,涉及政治、社科、经济、人文等诸多方面,其贯穿的核心应该是我自身对于社会和资本的思考。

作为当前的时代来说,法兰克福学派和利奥塔曾提出过“\textbf{资本主义可以吞噬一切}”,我对此赞同。当前的社科理论和现实均没有一个明确、完整、连贯的指向,全球、国家、组织,政治、经济、传媒、日常生活等均笼罩在被规训的迷雾中。专家学者往往也困在这迷雾中,或止步不前,或姑妄言之,或求得私利。

我并非有何大德大能,可以超越他人去给出一个精妙见解。但在这种种强力规诫中,作为社会原子个人虽受约束最多,限制和矛盾最多,但在某些方面却也\textbf{最为自由和活泼,最具生命力,规诫权力对此会出现失灵。}个人以求真、独立、自由的态度,与社会学习、对话、交流或许会有积极而强大的意义,即\textbf{个人的社会学}。

在这充满迷雾的利奥塔式“\textbf{独自漂流}”过程中,个人结合历史和现实,尽可能地学习、吸收、辩证、批判各路思想,去探寻社会前进的方法,并将个人的探寻结果回馈给社会。个人的所作所为可能激不起任何一点水花,远方也没有一个确定的彼岸;海上漂流的大部分时间是孤独的,只可与他人偶遇然后接着分离;所回馈社会的结果可能毫无影响力,也很可能是千疮百孔、一无是处;既无名声也无实利;却回报给个人精神气质上的昂扬和真诚。何曾有时——人类只能靠名声和金钱才能寻得一种快乐呢?

虽无彼岸,却拥有潘多拉魔盒中未曾逃逸出的“\textbf{希望}”,那就是社会的改良,伦理的进步。它可能过于浪漫主义空想,但是,在路上!在批判一切,包括批判自己寻求社会改良的道路上,必然得到\textbf{个人精神气质的昂扬和真诚}。我们努力,若有一天,我们倦了,不必强求,那就回航。其实我们所回去的港口无论如何也不会是我们出发的那一条港口了。

类似的思想,肯定有他人也在说,也在做。实在是不需再去寻找论文引证,这是理论上的自信,哈哈。真的需要去找篇论文来证明吗?即使回航,这也不是悲剧,我们的漂流便是我们的回报。这种“自下而上”的方式,或许真的有可能聚沙成塔呢。

我所写的电子书准备采用APACHE或GNU开源协议,源代码放置在\url{https://github.com/sd44/dingjia},已经编译好,尚在进行中的的草稿可于\textbf{草稿PDF文档}文件夹下下载。

虽是个人的社会学,我仍希望能尽可能的完善本书,使其更具意义。希望有人愿意帮我,或者完善文字,或者提供建议、指导,或者直接写入您的文章。

重申:\textbf{衷心希望各行各业人士同我交流},希望有合作者或者替代者能够加入这个项目,进行一系列的修正、改良或者重写!

我的Email:\href{mailto:sd44sd44@yeah.net}{sd44sd44@yeah.net}

QQ:25931014

微信:sd44sd44
