
\chapter{草稿之一——维基解密、阿桑奇与西方政治}

\section{维基解密简述}
维基解密官网的域名wikileaks.org注册于2006年10月4日,朱利安·保罗·阿桑奇一般被视为
其创始人。自维基解密成立之初,就着力于解密大批文档。创始之初采用公共编辑方式,任
何人都可以发布、修改页面。后来改为只接受具有政治、外交、历史或伦理意义的文件,所
提交文件需要匿名维基解密工作人员的审阅。

维基解密在十年多的时间里,多次对世界造成巨大影响。如公布肯尼亚原总统腐败案;阿尔
及利亚政府与石油公司合作,破坏另一家石油公司的设备造成石油大面积泄露;伊拉克战争
美军直升机射杀平民,包括两名路透社记者和儿童;阿富汗战争的大量文件及关塔那摩虐俘;
美国、俄罗斯的可以侵入几乎涵盖所有系统的网络黑客工具等。

\section{希拉里邮件门事件}
在2016年,维基解密先后发布10多万封有关于希拉里的邮件,一般被称为“邮件门”,影响
更是巨大。本节内容有借鉴以下网页。
\begin{enumerate}
\item \href{https://www.zhihu.com/question/41676600}{知乎:DNC邮件中有哪些美国民主党不可告人的内容?}

\item \href{https://www.zhihu.com/question/51362588}{知乎:如何看待The Podesta Emails?}

\item \href{https://github.com/zhouningyi/us_selection_crack}{GitHub:希拉里邮件
    门数据}
\end{enumerate}


为力求客观表述,克林顿基金会连续杀人、撒旦教披萨门等经由网友讨论演绎出的
阴谋论不计算在内,只节选其中极少数邮件和与之相关的后续发展如下。

2016年3月16日,维基解密发布30322封希拉里邮件,邮件收发时间跨度为2010年6月至2014年8月。

2016年7月22日,维基解密发布19252封DNC邮件,邮件收发时间跨度为2015年至2016年5月25日。
\begin{enumerate}
  \item emailid/25 希拉里竞选团队用邮件向DNC(民主党全国委员会)确认已收到几张DNC支
    票,被Jordan Kaplan严厉斥责:“不要再像这样发邮件。你认识Alex(直接跟他说)。
    不要犯蠢。”(根据网友分析,希拉里竞选团队通过HVF和DNC获取超过政治献金限额的
    捐赠,然后将超额捐赠化整为零,将这笔钱投入到广告或者通过小额筹款)
    
    \item emailid/1041 DNC中的Luis Miranda提供了造谣污损特朗普的几个方向。如特朗
      普危险、暴力、侮辱女性、穆斯林、墨西哥人、反对言论自由等。

      \item emailid/17065 富人Liz为HVF(希拉里胜利基金会)开具了一张20万美元的支票,要求希拉里参加在
        美国驻联合国人权理事会大使Eileen Donahoe家中举办的私人晚宴。
        
  \item emailid/20352 Jordan Kaplan要捐赠人名单,要求将名单发给Scott Comer。这些
    人有可能进入USPS, NEA, NEH等董事会,但也有可能进入不怎么好的董事会、理事会,
    如美国妇女历史委员会。

  \item emailid/658 Scott Comer提供了一个23名捐赠人名单。(据知乎帖 \url{https://www.zhihu.com/question/41676600},有部分人的捐赠数额,除去捐款最少的25美元和2600美元外,数额均在4万美元以上,最多捐款额为334000美元。)
\end{enumerate}

2016年10月7日,维基解密发布58000余封John Podesta的邮件。
Podesta于1998-2001年任比尔·克林顿的白宫幕僚长,2014-2015年任奥巴马总统顾
问,2015-2016年任希拉里竞选团队主席。
\begin{enumerate}
\item emailid/8396 2011年,卡塔尔邀请克林顿参加了纽约一个5分钟的小会,并承诺捐赠
给克林顿基金会100万美元,作为克林顿生日礼物。另外,卡塔尔对希拉里提出的海地投资
意见表示感谢,会加以考虑。

  \item emailid/7452 比尔·克林顿的幕僚长Tina Flournoy致信Podesta,外国政府捐赠
的钱已经入账。

  \item emailid/22030 摩洛哥提出向克林顿基金会支付1200万美元,但有条件——希拉里要
于2015年5月出席在摩洛哥古城马拉喀什为其召开的“克林顿全球倡议大会”,并在大会发
表演讲。(这在希拉里工作过的美国国务院已构成受贿行为,但希拉里仍然收下了这笔钱。
因担心影响选情后由其丈夫比尔·克林顿和女儿出席)

  \item emailid/6775 沙特谢赫穆罕默德酋长(可能翻译有误)想对克林顿提供飞机使其
参加埃塞俄比亚的会议一事表示感谢,要求克林顿亲自致电给他。Podesta同意Doug Band的
意见:除非穆罕默德酋长向克林顿基金会捐赠600万美元。

  \item emailid/4635 Podesta在疑与普京高度相关的Joule Unilimited公司持有75000股
股票。(后于奥巴马任期2014年时将股票转让给一家匿名控股私人公司)

  \item emailid/57027 民主党国家委员会的临时主席,也曾任职于CNN的Donna Brazile,
向希拉里泄露其与桑德斯党内辩论时要被问到的两个问题.在其他邮件中,Brazile想在希拉
里总统胜选后做Podesta的代理人。

\item emailid/39107 Alphabet公司(Google母公司)董事长Eric Emerson Schmidt的一个
  小团队为希拉里竞选团队制作竞选页面工具,并搜集整理捐赠者信息、信用卡号数据库等,
  团队工作人员认为Eric Emerson Schmidt暗示他可以做的更“全面”。(
  据
  \href{https://www.opensecrets.org/lobby/clientsum.php?id=D000067823&year=2017}{Center
    for Responsive Politics}统计,Alphabet/Google政治献金数额极大,2017年总游说费
  用为1815万美元,游说人数102人。)

\item emailid/8190 2008年10月6日,时任花旗银行高管Michael Froman发给Podesta一封主
  题为“Lists”的邮件,要求名单上的人应当优先考虑出任政府高官。附件分为3个,第一
  个是92个女性名单,第二个是222个非白种美国人及残疾美国人名单,第三个附件为31个内
  阁职位或内阁同级职位提名名单(附优先级与候选人)。(2008年大选投票日期是11月4日,
  奥巴马当选后,有近半数入选奥巴马内阁。)

  \item 其他,近百位媒体工作者、领导被Podesta招待。多位记者向Podesta表忠心,还有
    人预先告知将会提问希拉里的问题或者在发文章前请希拉里竞选团队过目。
\end{enumerate}


\section{维基解密的理念}

综观维基解密历史,它的理念应当是显而易见的。通过世人关注,吸收多渠道泄密出来的超
大批量文件,不管信息来自哪个渠道——黑客、政府和企业工作人员、某一派系的敌对方都可
以,只要解密文件是真实的便可接受。当信息量大到一定量级时,这个系统就是高度容错的
了,泄密者、甚至网站管理人员的个人主观就不起作用了。原本泄密相关的具体、特殊个人
和事件就会一个个结合起来,构成为更富普遍性、一般性、客观性的权力和金钱光谱。世人
由此看到这种光谱可怕的反伦理反人类,从而谴责、批评、改造、审判这种反伦理,号召权
力相关信息的更加公开化,以让世界变得更加阳光美好、公正平等、符合伦理。

以邮件门为例,\$illary Clinton和她"still dicking bimbos at home"的丈夫比尔·林顿本
来只是带有个人特色和性格的两个人,作为人来说他们是特殊的、个别的,在竞选中则代表
一方竞选势力。但随着邮件量级的增长,我们可以将其抽象到民主党,再到美国政治、资本
生态,再到西方,直至抽象到人类整个的——可以是当前的,也可以是历史的——权力、资本生态,
甚至抽象到人欲本身。

\section{维基解密的缺陷及潜在问题}

本小节内容框架主要来自于知乎用户\textbf{@阿迦陀},感谢\textbf{@阿伽陀}的批评指导。

有人指责维基解密与俄罗斯政府有关联,如DNC邮件泄露被怀疑有俄罗斯政府官员参与其中,
向维基解密递交了邮件。也有人指责维基解密与特朗普有关联。如阿桑奇就曾在某次电视节
目上自带“Vote Trump”的胸标,在个别采访也倾向于川普。但在一次采访中当主持人问他
选择投票给特朗普还是希拉里时,阿桑奇又说“霍乱还是淋病的选择吗?就我个人而言,我
一个都不喜欢。”。还有美国总统特朗普的长子小特朗普在推特上发布过这样一条信息:小
特朗普从2016年9月美国大选期间到今年7月之间与维基解密在Twitter上的私聊记录。其中一
条信息显示,维基解密希望小特朗普帮忙——让特朗普总统建议澳大利亚指派阿桑奇为澳驻美
大使。笔者个人认为阿桑奇向特朗普的倾斜出自现实的考虑,以使自己不必腹背受敌,谋求
大使职位则是希望借助大使的外交豁免权来保护阿桑奇不受伤害。

\begin{enumerate}
\item 不受制约的权力困境:

  维基解密试图借助公开秘密文件,打击不受制约和暗箱化操作的权力,使政治、军事、外
  交、伦理奔向更为阳光的一面。在维基解密的历史中,它也已具备极高可信度。但同时,
  维基解密的内容发布权限只集中在少数几个人的委员会,甚至阿桑奇自己手中。在试图制
  约其他权力的同时,维基解密的“发布委员会”自身也具备了极高的权力,这种权力由维
  基解密各种直接或间接的用户所赋予。

  那么,由谁来制约维基解密自身的权力呢?在现实中,因其倾向于特朗普,也在一定程度
  上影响了大选结果。勇者要谨防成为恶龙,而我们却不能提供预防恶变的措施,只能单薄
  无力地寄希望于阿桑奇自己的个人人格。

  比较容易想到的方案就是公众监督执行,但这样又会回到早期维基解密“公共编辑”方式,
  陷入信息大爆炸,从而难以筛选、管制、保持重心和尖锐,有价值的和无价值的、真实的
  和虚假的都混杂在一起。这种困境最终会使维基解密失去其“真实有力”的核心价值。

\item 核心机密文件的筹码困境:

  维基解密在因特网上分几次释放了数十GB加密文件,解密密码掌握在阿桑奇手中,数年来
  从未公开。这些大量加密文件被认为是足以造成国家动荡的核心机密,而用来解密这些文
  件的密码也是阿桑奇的“保命筹码”。凭此筹码,美国等国家不敢轻易直接对阿桑奇采取
  极端措施。维基解密掌握的这些最为有力的文件,在非极端情况下却几乎永不会为人所知。

\item 体制外权力的限度困境:

  维基解密是一种反体制规训的强大力量,这种力量无法融入体制。它寄希望于通过公众的
  知情权,以“自下而上”的方式来打破规训,走向美好,至于怎样“走向”,是它所不能
  提出的。也就是说,它具有极强批判性,但是建设性还是有限。

  @阿伽陀认为体制仍是重要且现实需要的,维基解密太过于反对体制,只能被反对派、寄生
  虫和别有用心者利用,我个人对此持有限赞同态度。我的观点阐述在本文最后两段。
\end{enumerate}


\section{希拉里邮件门之后}
(Warning: 本节还没写完,刚开始。)

希拉里邮件门,特别是Podesta邮件曝光后,美国媒体先是相当沉默,直至在公众网络上此事
越炒越热后才真正介入,CNN居然有主持人说公众直接去看泄密邮件是违法的,公众应当从媒
体、从CNN获取“权威”的泄密邮件相关报道。Quora,4Chan,Google均有删除维基解密相关
内容的行为。本是希拉里对立面的特朗普也往往避重就轻,顾左右而言他,皮尽管扯,里子
却是触动不得的。媒体、政治地被规训在此表现的淋漓尽致。

那维基解密理念中所在意的世界大众的力量呢?结果同样令人失望,表面看去并未起什么大
的波澜。短暂波动之后如此平静。这种平静让笔者害怕,害怕之余却又联想到历史。历史上
类似的政治腐败在曝光后,又有多少激起了民众非表面的、行动上带有积极意义的强烈反抗
呢?政治的核心,在现实实践上,是不是包括对统治阶级腐败的包容、淡漠与无视呢?如果
没有强力的、大范围的,甚至是暴力的反抗力量,公平正义平等是否是可行的呢?这种反抗
力量,是不是必然要具备理论上的和现实上的先进性才可以成立呢?

那么,我们是否可以说维基解密充其量是理想,而不具有现实的行动性呢?笔者认为不可以
这样认为。正因为有这些理想的火种散播在部分人心中,那就始终是有燎原希望的。倘若这
种理想火种因现实严酷不能很好发育,不能立刻引起改良就彻底否定它的意义,那么整个人
类社会都将持续向边沁、功利、利己的的方向发展。火种也是火,也是批判力量的雏形状态。
当潘多拉打开魔盒,放出一切邪恶与困病时,“希望”未来得及跑出,还留在里面。

正是因为我们的贫困现实(各方权力的规训力量日益强大,很可能无法提出一个行之有效的
总体建设方案),前进的道路只能是“自下而上”,寄托于公众权力的自我觉醒,这是一次
在浓雾中不见目的地的缓行。虽然缓慢迟缓,毕竟在应对日益增长的规训力量,否则公众只
能是被精神阉割,甘为鱼肉。

%%% Local Variables:
%%% mode: latex
%%% TeX-master: "../main"
%%% End:

