\chapter{削弱迷梦------论尼采的酒神和日神思想}
\label{chap:nicai}

在尼采的美学概念中,日神和酒神,即阿波罗(梦)和狄俄尼索斯(醉)两者均
为``\textbf{迷醉的类型}''\pagescite[][125]{ouxianghuanghun}。

\begin{quotation}
  按其词根来讲,阿波罗乃是``闪耀者、发光者'',是光明之神,他也掌管着内心幻想世界
  的\textbf{美的假象}。这种更高的真理,这些与无法完全理解的日常现实性相对立的状态
  的完满性,还有对在睡和梦中其治疗和帮助作用的自然的深度意识,同时也是预言能力的
  象征性类似物,一般地就是使生活变得可能,变得富有价值的各门艺术的象征性类似物。
  然而,有一条柔弱的界线,梦境不可逾越之,方不至于产生病态的作用,\textbf{不然的
    话,假象就会充当粗鄙的现实性来欺骗我们}。

  真实存在者和原始统一性,作为永恒受苦和充满矛盾的东西,为了自身得到永远的解脱,
  也需要\textbf{迷醉的解脱},也需要\textbf{迷醉的幻景}、\textbf{快乐的假象}。

  阿波罗以崇高的姿态向我们指出,这整个痛苦世界是多么必要,它能促使个体产生出具有
  解救作用的\textbf{幻景},然后使个体沉湎于幻景的关照中,安坐于大海中间一叶颠簸不
  息的小船上。\cite{beijudansheng}
\end{quotation}

《悲剧的诞生》一书的中译者孙国兴在译后记中总结到尼采所探求的问题是``人何以承受悲
苦人生''。尼采的答案是日神与酒神所融合的``悲剧——形而上学的慰藉'',即``变幻不
居的现象背后坚不可摧的、永恒的生命意志''。也就是尼采所说的``预感到太一怀抱中一种
至高的、艺术的原始快乐''。在这种形而上学意义上,``原始痛苦''与``原始快
乐''\textbf{根本是合一}的。

因现实的悲苦矛盾,每一人都不可避免、程度不一地受到日神阿波罗的影响,在方方面面产
生形态各异、程度不一的迷梦。但迷梦终归是不能长久或有力,过多的迷梦使人虚弱无力。
尼采所提出的方法是,削弱阿波罗式(时常作为``\textbf{守着种种界限和适度原则}''的、
被规训和限于迷梦的\textbf{个体化神化})的迷梦,正视现实的悲苦,而后以纵情、忘我的
酒神精神超越社会桎梏和个人,去实现生命和身体在原始期就具有的冲动创造意志,体
验生命的蓬勃。酒神精神使人回归到具有``\textbf{普遍性}''和``\textbf{至高意蕴}''的
生命和身体本身。

笔者所说个人社会学,部分受到尼采影响。个人社会学首先必然要去削弱强大现代性中日神
阿波罗的迷梦。其次它建立在人类全体这个普遍性之上——社会伦理(虽然尼采在一定程度上
反对社会伦理,但也鼓励对他个人的背离)。最后它本身不可避免带有\textbf{空想或试验
  的乌托邦性质}。从事个人社会学的个人,承担这种悲剧的同时也带有对悲剧的享受,感受
生命精神的勃发和昂扬。
