\chapter{向市场社会主义的转变1978-1992年}
\label{chap:1978}

\begin{enumerate}

\item 1977年11月15日,安徽省在时任省委第一书记万里的推动下下达了《关于当前农村经
  济政策几个问题的规定》(简称“省委六条”),有``尊重生产队的自主权''、``允许和
  鼓励社员经营正当的家庭副业,对收回的“自留地”,要按照政策规定如数退还给社
  员''等内容,安徽省再兴农业改革。1978年安徽遭遇大旱,省委决定允许借地给农民
  种“保命麦”,9月,肥西县山南公社黄花大队与山南公社馆西大队小井庄生产队借此政策
  搞起实质``包产到户'',12月凤阳县梨园公社严岗大队小岗生产队决定``包干到
  户''。1983年中央一号文件《当前农村经济政策的若干问题》,从理论、舆论、政策上确
  立和巩固了``实行生产责任制,特别是联产承包制;实行政社分设。''
  \footnote{\url{http://cpc.people.com.cn/GB/64162/64172/85037/85039/6619026.html}}

  需要注意的是,虽有省委推动,在党史上这块内容仍然被认为是农民主导。高王凌将农民
  主导的软性觉悟称为``反行为'',笔者对此知之甚少,但是``反行为''研究(不止农民)
  可能会在相当多的层面上有利于国计民生,希望有相关人士从事和深入此项工作。
  
  关于凤阳县``包干到户'',时至今日,仍有种思想认为它错误的将集体性质大幅削弱,笔
  者不认同这种意见。这种思想没有考虑当时中国实际情况,理论建设在一个空中楼阁之上。
  正如列宁在二十世纪初期就渴望资本强大、生产力发达的``美国式道路'',希望借此来达
  到俄国的资本发达,以此作为俄国向社会主义转变的物质基础,后来根据国情只能放弃这
  一提法,虽经后来斯大林、赫鲁晓夫等努力至苏联解体时苏联仍未真正实现``美国式道
  路''一样,理论宏观上是能自洽的,实践中却有太多微观细节和问题要面对。

  ``统购统销''政策需要国家有实力去吸纳农村产出以支持``工农业剪刀差''的正常运作:
  国家以相对“高价格”收购农村农副产品,补贴城镇职工,使城镇职工可以保有相对低的
  工资。(这种做法在马克思主义理论中其实是提高相对剩余价值。)同时,国家也可在自
  身维持的农副产品价格稳定状态中获取相对于自然经济价格的利差去发展工业,毛时代尤
  其侧重重工业,邓时代减弱了重工业的扶持力度。但刚经历完多年动荡的我国已经没有力
  气再去做这件事了。实际上根据薄一波所说,统购统销造成的财政缺口是巨大的:
  \begin{quotation}
    (统购统销)第二年(即1954至1955年度),即赔了2.5506亿元。随着粮食经营费用的
    增加和购销价格``倒挂''现象的出现,亏损越来越大……1987年达276亿元,1988年突
    破300亿元,成为国家财政的一大包袱。\pagescite[][281]{boyibo}
  \end{quotation}

  即使是``包干到户''这种落后小农经济的生产力、生产关系,国家在1984年这个``特大丰
  收年''已经无法去完成大批量的``统购统销''了。1985年,国家改行``合同订购''制度并
  在年底要求对订购数量逐年降低,统购统销的精神纲领仍在,并且``合同订购''给的农产
  品价格更低。\cite{diyibugaige}

  \begin{quotation}
    根据农业经济专家\footnote{汪晖此处所指专家为曾任国务院农村发展研究中心发展研
      究所市场研究室主任等职务的卢迈}的研究,1978~1985年城乡收入的差距是缩小的,
    从1985年起扩大。1989年到1991年农民收入增长基本停滞,城乡收入差距又恢复
    到1978年以前的情况。1993年以后,由于国家提高粮食价格、乡镇企业增长快、外出务
    工人口收入增长等原因,农村收入增长较快,但在城市劳动力大量剩余的情况下,这一
    势头正在改变。\cite{wangxiandai}
  \end{quotation}

  三农问题和工农业剪刀差是一个实践上的大难题,在数年内长期困扰着中国。即使幻想国
  家在1980年后不惜损害工业、全球竞争力等一系列代价抽调其他方面资金、资料用来补贴
  农业生产发展,那么农业生产力提高又将带来时代环境下所不能解决的大量剩余劳动力问
  题——城市和农村均无法吸收。

  1992年底,统购统销正式退出历史舞台。而到了今日,我们已经开始了农业工业化的道路。
  我们放在文后再讨论这两个节点吧。

\item 1978年4月5日,中共中央批转公安部《关于全部摘掉“右派”分子帽子的请示报告
  》,``到1981年,在全面复查的基础上,对错划右派的3改正和落实政策工作全部完成。全
  国共改正错划右派54万人,占原划右派总数
  的98\%''\url{http://www.zgdsw.org.cn/n/2013/0125/c244520-20323571.html}。

\item 1978年5月11日,《光明日报》发表胡福明原作后经多次修订的《实践是检验真理的唯
  一标准》,新华社转发。次日《人民日报》和《解放军报》同时转载,拉开了全国范围内
  极左路线失势的序幕,华国锋``两个凡是''理论倒台。

  笔者认为需要说明的是,不考虑本文背后的社会和政治意义,其中的理论本身是正确的,
  也是马克思一直坚持的,他所关注的正是将主观和客观相互作用联系起来的实践中的人和
  社会,正是这种实践的要求,要求理论与时俱进,要求马克思主义理论对其自身的批判和
  发展。也正因此,马克思抛弃了形而上学的终极实存的哲学体系,也同时抛弃了纯粹的、
  试图超越人类历史和实践的自然辩证法,他所提倡的是成为运动的、实践的、历史的、辩
  证的、人的历史唯物主义,并试图改造世界以此让世界更加美好。笔者再次重申,马克思
  历史唯物主义事实上批判了科学社会主义,详情请见\namecref{sec:marxkexue}一节。

\item 1978年11月10日--12月15日,在北京召开中央工作会议,原定议题讨论经济。陈云在
  东北组率先提出系统地解决历史遗留问题,平凡一系列重大冤假错案。(“到1982年底,
  约有300多万名干部得到平反。”)邓小平提出``再不实行改革,我们的现代化事业和社会
  主义事业就会被葬
  送''\footnote{\url{http://cpc.people.com.cn/GB/64184/64190/65724/4444934.html}},胡耀邦``认为有些农村体制如“政社合一”就应该改变。''
  1978年12月13日,邓小平总结发言《解放思想,实事求是,团结一致向前看》,提到反官
  僚主义、加强责任制、``允许一部分地区、一部分企业、一部分工人农民,由于辛勤努力
  成绩大而收入先多一些,生活先好起来'',同时对落后地区和人民给以有力支持等等。
  
  1978年12月18日至22日,党的十一届三中全会在北京召开,继承了之前中央工作会议精神,
  在全中国及历史层面上成为极其重要的一个转折点,另外``邓小平实际上已成为中央领导
  集体的核心。''。\footnote{\url{http://cpc.people.com.cn/GB/64184/64190/65724/4444934.html}}

\item 1980年,中央设立

\item 1980年8月18日,邓小平在中共中央政治局扩大会议上作了《党和国家领导制度的改革》
  的讲话,提出干部队伍要年轻化、知识化、专业化,要建立退休制
  度\url{http://news.12371.cn/2017/03/09/ARTI1489054449475294.shtml}。

  对中国政治体制中的“权力过分集中”的弊端进行了严厉批判,首次提出了“党和国家领导制度改革”的问题。这篇讲话,后来被中共十三大尊为“中国政治体制改革的纲领性文献”,也被党内外的主流研究者们奉为研究邓小平政治体制改革思想的经典。






位。\footnote{\url{http://www.people.com.cn/GB/historic/0510/6267.html}}

从1978-1992年是中国左右思想交错争锋的一段时间。\improve{未完}

\end{enumerate}





% ( 4) 共产主义社会高级阶段。 中国共产党第十三次全国代表大会的报告中指出, 社会主义初级阶
% 段 “ 不是泛指任何国家进入社会主义社会都会经历的起始阶段, 而是特指我国在生产力落后、 商品
% 经济不发达条件下建设社会主义必然要经历的特定阶段 。” [6 ]
%  发达资本主义国家在无产阶级夺取   出自《正确认识和对待“共产主义渺茫论”》




%%% Local Variables:
%%% mode: latex
%%% TeX-master: "../main"
%%% End:
